\documentclass[10pt,a4paper]{article}
\usepackage[utf8]{inputenc}
\usepackage{amsmath}
\usepackage{amsfonts}
\usepackage{amssymb}
\begin{document}

\noindent Andrea Gadotti \hfill 18 marzo 2014 - Revisione 0 

\

\section*{Teoria dei Modelli - Primo foglio di esercizi}

\
\subsection*{Esercizio 2.20}
La formula $\psi= \forall x y \, (r(x,y) \wedge r(u,v) \Rightarrow \neg r(x,u) \wedge \neg r(y,v))$ soddisfa le richieste dell'esercizio. La verifica è diretta e immediata per entrambi i versi dell'implicazione.

\subsection*{Esercizio 3.43}
Per trovare $K$, è sufficiente ripetere la dimostrazione del teorema di Löwenheim-Skolem, con la seguente modifica: per assicurarci che venga soddisfatta anche la richiesta $K \cap M \preceq N$, è necessario imporre che, nel caso in cui la formula $\varphi_k(x)$ contenga solo parametri in $M$ (oppure nessun parametro), allora anche $a_k$ appartiene ad $M$. Questo è sempre possibile, in quanto per ipotesi abbiamo che $M \preceq N$.\\
La verifica che $K \preceq N$ è identica a quella della dimostrazione originale. Anche la verifica che $K \cap M \preceq N$ è analoga; infatti, $K \cap M$ soddisfa le ipotesi del criterio di Tarski-Vaught, poiché le formule esistenziali a parametri in $K \cap M$ possiedono testimoni in $K \cap M$ per costruzione.


\end{document}