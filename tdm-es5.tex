\documentclass[10pt,a4paper]{article}
\usepackage[utf8]{inputenc}
\usepackage{amsmath}
\usepackage{amsfonts}
\usepackage{amssymb}
\usepackage[italian]{babel}
\usepackage{euscript}

\def\ccl{\textrm{ccl}}
\def\st{\textrm{st}}
\def\phi{\varphi}
\def\D{\EuScript D}
\def\Ee{\EuScript E}
\def\P{\EuScript P}
\def\K{\EuScript K}
\def\U{\EuScript U}
\def\ZZ{\mathds Z}
\def\NN{\mathds N}
\def\QQ{\mathds Q}
\def\RR{\mathds R}
\def\<{\langle}
\def\>{\rangle}
\def\E{\exists}
\def\A{\forall}
\def\0{\varnothing}
\def\equivL{\stackrel{\smash{\scalebox{.5}{\rm L}}}{\equiv}}
\def\imp{\rightarrow}
\def\iff{\leftrightarrow}
\def\IMP{\Rightarrow}
\def\IFF{\Leftrightarrow}
\def\range{\textrm{im}}
\def\Aut{{\rm Aut}}
\def\tp{{\rm tp}}
\def\acl{{\rm acl}}
\def\eq{{\rm eq}}
\def\Mod{\textrm{Mod}}
\def\models{\vDash}
\def\proves{\vdash}
\def\notmodels{\nvDash}
\def\sm{\smallsetminus}
\def\ssf#1{\textsf{\small #1}}
\DeclareMathOperator{\dom}{dom}
\begin{document}

\noindent Andrea Gadotti \hfill 27 aprile 2014 - Revisione 0 

\

\section*{Teoria dei Modelli - Quinto foglio di esercizi}

\

\subsection*{Esercizio 1}
Lavoriamo all'interno di un modello $\U$ saturo e di cardinalità grande. Si dimostri che per ogni $A\subseteq N$ esiste un $M$ tale che $\acl A\,=\,M\cap N$.\\
\\
\noindent\textbf{Soluzione} Sia $\lambda:=|L|+|A|+\omega$. Riprendiamo lo schema della dimostrazione del teorema di Löwenheim-Skolem all'ingiù, la versione in 3.41. Scegliamo questa versione della dimostrazione in quanto avremo bisogno di aggiungere un solo elemento ad ogni passo.\\
Procediamo per induzione transfinita come in 3.41, definendo $A_0:=A$. Per il passo induttivo, richiediamo anche che $\acl(A_i)\cap N\subseteq\acl(A)$. Fissiamo una variabile $x$ ed enumeriamo tutte le formule $\langle \phi_k(x): k < \lambda \rangle$ a parametri in $A_i$ che sono consistenti in $\U$. Ora, sia $i=\<i_1,i_2\>$, in una fissata enumerazione di $\lambda^2$  di lunghezza $\lambda$ e scegliamo $b$ soluzione della $i_1$-esima formula a parametri in $A_{i_2}$ tale che $\acl(A_i,b)\cap N\subseteq\acl(A)$ (dimostreremo alla fine che un tale $b$ esiste). Definiamo $A_{i+1}=A_i\cup\{b\}$. Definiamo $M$ come l'unione degli $A_i$. Usando il criterio di Tarski-Vaught si vede subito che $M \preceq \U$. Inoltre, per costruzione, $\acl(M) \cap N = M \cap N \subseteq \acl(A)$. L'inclusione $\acl(A) \subseteq M \cap N$ è banale perché $M$ e $N$ sono due modelli che contengono $A$.\\
Dimostriamo ora che, sotto l'ipotesi induttiva, esiste un $b$ come richiesto. Si vede facilmente che la richiesta su $b$ equivale a chiedere che $b$ realizzi il tipo $p(x)$ così definito (scriviamo $\phi$ al posto di $\phi_{i_1}$):
$$\big\{\phi(x)\big\} \ \cup \ \big\{\psi(b,x)\imp\neg\E^{\le n}y\,\psi(y,x)\ :\ b\in N\sm\acl(A),\ \ \ \psi(y,x)\in L(A_i),\ \ \ n\le\omega\big\}$$

Occorre mostrare quindi che $p(x)$ è consistente in $\U$. Supponiamo per assurdo che non lo sia. Allora per saturazione non è nemmeno finitamente consistente. Questo significa che esiste (un sottoinsieme finito di $\omega$, o equivalentemente, nel nostro caso, esiste) $\overline{n} \in \omega$, ed esistono un numero finito di formule $\psi_j(y,x)$ a parametri in $A_i$, $1 \leq j \leq m$ per qualche $m$ naturale, e per ogni $j$ esistono un numero finito di $b_{j,k} \in N \sm \acl(A)$, $1 \leq k \leq p_j$ per qualche $p_j$ naturale, tali che per ogni $x \in \U$ vale 

\begin{equation*}
\begin{split}
\neg \phi(x)& 	\vee [\psi_1(b_{1,1},x) \wedge \E^{\le n}y\,\psi_1(y,x)] \vee ... \vee [\psi_1(b_{1,p_1},x) \wedge \E^{\le n}y\,\psi_1(y,x)] \\
			&	\vee [\psi_2(b_{2,1},x) \wedge \E^{\le n}y\,\psi_2(y,x)] \vee ... \vee [\psi_2(b_{2,p_2},x) \wedge \E^{\le n}y\,\psi_2(y,x)] \\
			&	\vdots \\
			&	\vee [\psi_m(b_{m,1},x) \wedge \E^{\le n}y\,\psi_m(y,x)] \vee ... \vee [\psi_m(b_{m,p_m},x) \wedge \E^{\le n}y\,\psi_m(y,x)] \\
\end{split}
\end{equation*}

\noindent \emph{NOTA: vorrei dimostrare che questo è assurdo. Purtroppo non riesco a farlo, e in realtà ho anche qualche dubbio sul fatto che l'assurdo ci sia davvero. Infatti, riesco a dimostrare l'assurdo solo nel caso in cui l'insieme dei $b_{j,p_j}$ è formato da un solo elemento $b$. Procedo quindi con la dimostrazione assumendo questo fatto per ipotesi, sperando che la dimostrazione sia nel complesso corretta e ci sia da sistemare solo questo problema.}\\
\\
Supponiamo allora che per ogni $x \in \U$ valga
$$\neg \phi(x) \; \vee \; [\psi_1(b,x) \wedge \E^{\le n}y\,\psi_1(y,x)] \vee ... \vee [\psi_m(b,x) \wedge \E^{\le n}y\,\psi_m(y,x)]$$

Sia $q$ una qualsiasi soluzione di $\phi(x)$ (che esiste perché $\phi(x)$ consistente in $\U$ per ipotesi). Allora deve valere $\psi_j(b,q) \wedge \E^{\le n}y\,\psi_j(y,q)$ per qualche $j$. Ovvero $b$ è algebrico su $A_i \cup \{q\}$, i.e. $b \in \acl(A_i,q)$ per ogni $q \models \phi(x)$. Ma allora per l'esercizio 10.10 abbiamo che $b \in acl(A_i)$. Ma per ipotesi $b \in N \sm \acl(A)$, il che contraddice l'ipotesi induttiva.

\subsection*{Esercizio 2}

Sia $T$ una teoria fortemente minimale. Siano $M\preceq N$ tali che $\dim N=\dim M+1$. Si dimostri che non esiste un modello $K$ tale che $M\prec K\prec N$.\\
\\
\noindent\textbf{Soluzione} Supponiamo per assurdo che esista $K$ modello tale che $M\prec K\prec N$. Allora $\exists a \in K \sm M$ e $\exists b \in N \sm K$. Sia $B_M$ una base per $M$, che esiste per 10.16 ($T$ fortemente minimale per ipotesi). Allora, poiché $\acl(B_m) = M$, si ha che $\acl(B_m) \cup \{a\}$ è indipendente per 10.14. Dato che $B_m \cup \{a\} \subseteq K$, allora $\acl(B_m \cup \{a\}) \subseteq \acl(K)=K$. Quindi $\dim K \geq \dim M +1$.\\
Ripetendo lo stesso ragionamento con $K$, $N$ e $b$ otteniamo $\dim N \geq \dim K +1 \geq \dim M +2$, contro le ipotesi.


\subsection*{Esercizio 3}
Sia $T$ una teoria fortemente minimale. Si dimostri che ogni insieme infinito algebricamente chiuso è un modello.\\
\\
\noindent\textbf{Soluzione} Utilizziamo il criterio di Tarski-Vaught. Sia $\phi(x)$ una formula a parametri in $A$ consistente in $\U$. Se $\phi(x)$ è algebrica, allora è soddisfatta da qualche $a \in \acl(A)=A$. Se invece $\phi(x)$ non è algebrica, allora definisce un insieme infinito. Poiché $T$ è minimale, questo insieme deve essere cofinito. Ma poiché $A$ è infinito, $A$ deve necessariamente contenere una soluzione di $\phi(x)$.



\end{document}